\documentclass[longbibliography,nofootinbib,twocolumn]{revtex4-1}

\newcommand{\network}{The NuCypher Network}

\usepackage{listings}
\usepackage{graphicx}
\usepackage{amsmath}
\usepackage[margin=5pt]{subfig}
\usepackage[usenames]{color}

\renewcommand{\baselinestretch}{1.4}
\setlength{\parskip}{1em}
\definecolor{darkgreen}{rgb}{0.00,0.50,0.25}
\definecolor{darkblue}{rgb}{0.00,0.00,0.67}
\newcommand{\figref}[1]{Fig.~\ref{#1}}
\usepackage[breaklinks,pdftitle={The NuCypher Network: A decentralized
cryptological network offering accessible, intuitive, and extensible runtimes and interfaces for secrets management and dynamic access
control}, pdfauthor={NuCypher},colorlinks,urlcolor=blue,citecolor=darkgreen,linkcolor=darkblue]{hyperref}
\graphicspath{{pdf/}}

\usepackage[T1]{fontenc}
\usepackage{lmodern}
\lstset{
    basicstyle=\ttfamily,
    basewidth={0.5em, 0.5em},
    columns=fullflexible,
}

\begin{document}

\title{\network: A decentralized cryptological network offering accessible, intuitive, and extensible runtimes and interfaces for secrets management and dynamic access control}

\author{Michael Egorov}
\email{michael@nucypher.com}
\author{David Nu{\~n}ez}
\email{david@nucypher.com}
\author{MacLane Wilkison}
\email{maclane@nucypher.com}
\affiliation{NuCypher}

\begin{abstract}
    \network~provides accessible, intuitive, and extensible runtimes and interfaces for secrets management and dynamic access control.
    It's accessible by virtue of being decentralized, permissionless, and censorship-resistant: there are no gate-keepers and anyone can use it.
    It's intuitive thanks to the classic character-based narrative of Alice and Bob (with the introduction of additional cryptological characters
    where appropriate), that permeates the code-base and helps developers write safe, misuse-resistant code.
    It's extensible as it currently supports proxy re-encryption but can be extended to provide support other cryptographic primitives.

    \network~enables developers to manage secrets and dynamically grant and revoke access to sensitive data in public networks.
\end{abstract}

\date{\today}
\maketitle

\tableofcontents

\section{Introduction}

\section{Network}

\section{Smart contract layer}

\section{NU token and staking economics}

\section{Characters}

\section{Conclusion}

\bibliography{whitepaper}

\end{document}
